\documentclass[a4paper, 11pt]{article}

% Classical useful packages
\usepackage[french]{babel}
\usepackage[utf8]{inputenc}  
\usepackage[T1]{fontenc} 
\usepackage{multicol}
 
% For math
\usepackage{amsmath,amsthm,amsfonts,amssymb}
\usepackage{upgreek}

% For item blocks
\usepackage{multienum,enumerate,enumitem}

%Color
\usepackage{color}
\usepackage{xcolor}

%hyper link
\usepackage[backref=page]{hyperref}

  
\hypersetup{
	plainpages=false,
	unicode=true,          % non-Latin characters in Acrobat\u2019s bookmarks
	pdftoolbar=true,        % show Acrobat\u2019s toolbar?
	pdfmenubar=true,        % show Acrobat\u2019s menu?
	pdffitwindow=true,      % page fit to window when opened
	pdftitle={},    % title
	pdfauthor={Author},     % author
	pdfsubject={Subject},   % subject of the document
	pdfcreator={Creator},   % creator of the document
	pdfproducer={Producer}, % producer of the document
	pdfkeywords={keywords}, % list of keywords
	pdfnewwindow=true,      % links in new window
	colorlinks=true,       % false: boxed links; true: colored links   %%%%%%%%%%% MODIFIE
	linkcolor=black,          % color of internal links
	citecolor=black,        % color of links to bibliography
	filecolor=cyan,      % color of file links
	urlcolor=magenta,          % color of external links  
	urlbordercolor=0 1 1
	}
\usepackage{listings} 
\lstset{ 
	language=Matlab,
	basicstyle=\small\ttfamily,
	columns=fullflexible,
	breaklines=true,
	keywordstyle=\color{blue},
 	commentstyle=\bfseries\color{green!40!black},
	frame=single,
	numbers=left,
	numberstyle=\small\ttfamily,
}

%For euro symbol: no needed here
\usepackage{esint}

%For bitmap style and vectorial police
\usepackage{lmodern}

%Tikz 
\usepackage{pgf, tikz}

%For fancy boxes
\usepackage{fancybox} %\Ovalbox{text}

%Geometry of the doc, limiter of pages 
\usepackage{geometry}
\geometry{
paperwidth=21cm,
left=2.2cm,
right=2.2 cm,
paperheight=29.7cm,
top=1.9cm,
height=26.5cm,
footskip = 1cm
}

\usepackage{palatino}

\usepackage{graphicx}
\graphicspath{{./}{./chapters/}{./images/}}

\usepackage{dsfont}
\usepackage[titletoc]{appendix} % Appendix

\usepackage{mathtools}
\numberwithin{equation}{section}

%set
\def\R{\mathbb{R}}
% \def\A{\mathbb{A}}
% \def\B{\mathbb{B}}
\def\N{\mathbb{N}}
% \def\T{\mathbb{T}}
% \def\G{\mathbb{G}}
% \def\D{\mathbb{D}}
\def\I{\mathbb{I}}
\def\K{\mathbb{K}}
\def\C{\mathbb{C}}
\def\Z{\mathbb{Z}}
\def\E{\mathbb{E}}

%differential operator

%\def\d{\mathrm{div}}&
\def\div{\operatorname{div}}

% Hausdorff and Lesbegue measures 

\def\H{\mathcal{H}}
\def\L{\mathcal{L}}

% Operators 
\def\X{\mathcal{X}}
\def\Y{\mathcal{Y}}
\def\P{\mathcal{P}}
\def\D{\mathcal{D}}
\def\Re{\mathcal{R}}
\def\DR{\mathcal{DR}}
\def\RD{\mathcal{RD}}
\def\S{\mathcal{S}}
\def\A{\mathcal{A}}
\def\B{\mathcal{B}}
\def\G{\mathcal{G}}
\def\T{\mathcal{T}}
\def\W{\mathcal{W}}
\def\Co{\mathcal{C}}
\def\V{\mathcal{V}}

\def\relu{\mathrm{ReLU}}

% Boolean and measures spaces 

\def\B{\mathcal{B}}
\def\M{\mathcal{M}}
\def\BV{\mathrm{BV}}

% Fourier transform 
\def\F{\mathcal{F}}

% asymptotics
\def\O{\mathcal{O}}


% argmin

\DeclareMathOperator*{\argmin}{argmin}



\DeclarePairedDelimiter{\norm}{\lVert}{\rVert} %% requires mathtools
\DeclarePairedDelimiter{\abs}{\lvert}{\rvert}

% for subordinate norm 
\newcommand{\vertiii}[1]{{\left\vert\kern-0.25ex\left\vert\kern-0.25ex\left\vert #1 
    \right\vert\kern-0.25ex\right\vert\kern-0.25ex\right\vert}}

\makeatletter
% boldmath in titles
\g@addto@macro\bfseries{\boldmath}
% straight d in integrals
\renewcommand\d[1]{\mspace{2mu}\mathrm{d}#1\@ifnextchar\d{\mspace{-1mu}}{}}
% Swap the definition of \abs* and \norm*, so that \abs
% and \norm resizes the size of the brackets, and the 
% starred version does not.
\let\oldabs\abs
\def\abs{\@ifstar{\oldabs}{\oldabs*}}
%
\let\oldnorm\norm
\def\norm{\@ifstar{\oldnorm}{\oldnorm*}}
\makeatother

\def\disp{\displaystyle}

% box for theorem, remark ...
\newtheorem{thm}{Théorème}[section]
\newtheorem{prop}[thm]{Proposition}
\newtheorem{lem}[thm]{Lemme}
\theoremstyle{definition}
\newtheorem{definition}[thm]{Définition}
\newtheorem{cor}[thm]{Corollaire}

\newtheorem{claim}[thm]{Claim}
\theoremstyle{remark}
\newtheorem{rem}[thm]{Remarque}
%\newtheorem{remf}[thm]{Remarque}
\newtheorem{ex}[thm]{Exemple}
\newtheorem{exo}[thm]{Exercice}


% time step 
\def\dt{\delta_t}
\usepackage{blindtext}


%\renewcommand\thechapter{\Roman{chapter}}
\renewcommand{\thesection}{\arabic{section}}
%\renewcommand{\thesection}{\Roman{section}}

\renewcommand\qedsymbol{$\blacksquare$}



%---- Style de l'entete -----
\setlength{\parindent}{0cm}
\newcommand{\entete}
{
% Lieu - annee
%  \noindent {\textbf{Université de la Polynésie Française} \hfill \textbf{Semestre 5 ~ 2023-2024}}\\
  %  Module
%  \noindent {L3 Math \hfill UE : Optimisation\\}
  % Titre
  \hrule
  \begin{center} \textbf{\Large Most common Git commands} \\ \textsc{List descrition} \end{center}
  \hrule
  \vspace*{0.3cm}
}
  
\begin{document}
\entete

\section{Initializing a Git Repository}
\begin{itemize}[label=$\bullet$]
    \item \textbf{git init:} Initialize a new Git repository in the current directory.
\end{itemize}

\section{Staging and Committing Changes}
\begin{itemize}[label=$\bullet$]
    \item \textbf{git add <file>:} Stage a file for commit.
    \item \textbf{git add .:} Stage all changes for commit.
    \item \textbf{git commit -m "Commit message":} Commit changes with a message.
\end{itemize}

\section{Checking the Status of a Repository}
\begin{itemize}[label=$\bullet$]
    \item \textbf{git status:} Show the current status of all files in the repository.
    \item \textbf{git diff:} Show differences between the working directory and the last commit.
\end{itemize}

\section{Configuring Git}
\begin{itemize}[label=$\bullet$]
    \item \textbf{git config user.name "Your Name":} Set your name for commits.
    \item \textbf{git config user.email "youremail@example.com":} Set your email for commits.
\end{itemize}

\section{Working with Branches}
\begin{itemize}[label=$\bullet$]
    \item \textbf{git branch:} List all branches.
    \item \textbf{git branch <branch name>:} Create a new branch.
    \item \textbf{git checkout <branche name>:} Switch to a branch.
    \item \textbf{git merge <branche name>:} Merge a branch into the current branch.
    \item \textbf{git branch -d <branche name>:} Delete a branch.
\end{itemize}

\section{Connecting to a Remote Repository}
\begin{itemize}[label=$\bullet$]
    \item \textbf{git remote add <remote name> <remote url>:} Add a remote repository.
    \item \textbf{git clone <remote url>:} Clone a remote repository to a local directory.
    \item \textbf{git pull <remote name> <branche name>:} Get the latest changes from a remote repository.
    \item \textbf{git push <remote name> <branche name>:} Send local commits to a remote repository.
\end{itemize}

\section{Getting and Sending Changes}
\begin{itemize}[label=$\bullet$]
    \item \textbf{git pull:} Get the latest changes from the remote repository.
    \item \textbf{git push:} Send local commits to the remote repository.
\end{itemize}

\section{Stashing Changes}
\begin{itemize}[label=$\bullet$]
    \item \textbf{git stash:} Save changes made when they're not in a state to commit them to a repository.
    \item \textbf{git stash pop:} Apply the most recent stash and remove it from the stack.
\end{itemize}


\end{document}
